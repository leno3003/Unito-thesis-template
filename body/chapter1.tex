\chapter[State of The Art]{State of The Art}
\section{DeepFaceLab}
\cite{dfl} DeepFaceLab is the current dominant deep-fake
framework for face-swapping. It provides the necessary tools 
as well as an easy-to-use way to conduct high-quality face-swapping. 


Face swapping is an eye-catching task in generating
fake content by transferring a source face to the destination
while maintaining the destination’s facial movements and
expression deformations.


It is noteworthy that DFL falls in a typical one-to-one 
face-swapping paradigm, which means there
are only two kinds of data: src and dst, the abbreviation for
source and destination.


In order to achieve high-quality face swaps, GANs (Generative
Adversarial Networks) are used.
DeepFaceLab provides a set of workflow which form a
flexible pipeline. In DeepFaceLab, we can
abstract the pipeline into three phases: \textbf{extraction},
\textbf{training}, and \textbf{conversion}.

\textbf{Extraction:} The extraction phase is the first phase in DFL, aiming to
extract a face from src and dst data. This phase consists
of many algorithms and processing parts, i.e., face detection, face alignment, and face segmentation.

\textbf{Training:} DeepFaceLab's structure consists of an
Encoder as well as Inter with shared weights between src
and dst, two Decoders which belong to src and dst separately. 

\textbf{Conversion:} 
The first step of the proposed face-swapping scheme in 
the conversion phase is to transform the
generated face alongside with its mask from dst Decoder
to the original position of the target image in src.
For the sake of remaining consistent complexion, 
DFL provides five more color
transfer algorithms (i.e. RCT, IDT, \dots) to approximate the 
color of the reenacted face to the target.
\newpage
\section{Encoder-Decoder}
\section{Stylegan2}
\section{ThisPersonDoesNotExist}
\section{GANs}
Generative Adversarial Networks\cite{AICI, UGAN, DPG}, are a class of machine
learning frameworks in which, two neural networks (a generative network
and a discriminative network) contest each other in a game (a sum zero
game).
Given a training set, this tecnique learns to generate new data with the
same statistics as the training set.
The generative network generates candidates while discriminative network
evaluates them. The generative network’s training objective is to incre
the error rate of the discriminative network (pratically, it tries to fool the
discriminative network).
